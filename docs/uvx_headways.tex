\documentclass[]{elsarticle} %review=doublespace preprint=single 5p=2 column
%%% Begin My package additions %%%%%%%%%%%%%%%%%%%
\usepackage[hyphens]{url}

  \journal{Submitted to Transport Findings} % Sets Journal name


\usepackage{lineno} % add
\providecommand{\tightlist}{%
  \setlength{\itemsep}{0pt}\setlength{\parskip}{0pt}}

\usepackage{graphicx}
\usepackage{booktabs} % book-quality tables
%%%%%%%%%%%%%%%% end my additions to header

\usepackage[T1]{fontenc}
\usepackage{lmodern}
\usepackage{amssymb,amsmath}
\usepackage{ifxetex,ifluatex}
\usepackage{fixltx2e} % provides \textsubscript
% use upquote if available, for straight quotes in verbatim environments
\IfFileExists{upquote.sty}{\usepackage{upquote}}{}
\ifnum 0\ifxetex 1\fi\ifluatex 1\fi=0 % if pdftex
  \usepackage[utf8]{inputenc}
\else % if luatex or xelatex
  \usepackage{fontspec}
  \ifxetex
    \usepackage{xltxtra,xunicode}
  \fi
  \defaultfontfeatures{Mapping=tex-text,Scale=MatchLowercase}
  \newcommand{\euro}{€}
\fi
% use microtype if available
\IfFileExists{microtype.sty}{\usepackage{microtype}}{}
\usepackage{natbib}
\bibliographystyle{plainnat}
\usepackage{color}
\usepackage{fancyvrb}
\newcommand{\VerbBar}{|}
\newcommand{\VERB}{\Verb[commandchars=\\\{\}]}
\DefineVerbatimEnvironment{Highlighting}{Verbatim}{commandchars=\\\{\}}
% Add ',fontsize=\small' for more characters per line
\usepackage{framed}
\definecolor{shadecolor}{RGB}{248,248,248}
\newenvironment{Shaded}{\begin{snugshade}}{\end{snugshade}}
\newcommand{\AlertTok}[1]{\textcolor[rgb]{0.94,0.16,0.16}{#1}}
\newcommand{\AnnotationTok}[1]{\textcolor[rgb]{0.56,0.35,0.01}{\textbf{\textit{#1}}}}
\newcommand{\AttributeTok}[1]{\textcolor[rgb]{0.77,0.63,0.00}{#1}}
\newcommand{\BaseNTok}[1]{\textcolor[rgb]{0.00,0.00,0.81}{#1}}
\newcommand{\BuiltInTok}[1]{#1}
\newcommand{\CharTok}[1]{\textcolor[rgb]{0.31,0.60,0.02}{#1}}
\newcommand{\CommentTok}[1]{\textcolor[rgb]{0.56,0.35,0.01}{\textit{#1}}}
\newcommand{\CommentVarTok}[1]{\textcolor[rgb]{0.56,0.35,0.01}{\textbf{\textit{#1}}}}
\newcommand{\ConstantTok}[1]{\textcolor[rgb]{0.00,0.00,0.00}{#1}}
\newcommand{\ControlFlowTok}[1]{\textcolor[rgb]{0.13,0.29,0.53}{\textbf{#1}}}
\newcommand{\DataTypeTok}[1]{\textcolor[rgb]{0.13,0.29,0.53}{#1}}
\newcommand{\DecValTok}[1]{\textcolor[rgb]{0.00,0.00,0.81}{#1}}
\newcommand{\DocumentationTok}[1]{\textcolor[rgb]{0.56,0.35,0.01}{\textbf{\textit{#1}}}}
\newcommand{\ErrorTok}[1]{\textcolor[rgb]{0.64,0.00,0.00}{\textbf{#1}}}
\newcommand{\ExtensionTok}[1]{#1}
\newcommand{\FloatTok}[1]{\textcolor[rgb]{0.00,0.00,0.81}{#1}}
\newcommand{\FunctionTok}[1]{\textcolor[rgb]{0.00,0.00,0.00}{#1}}
\newcommand{\ImportTok}[1]{#1}
\newcommand{\InformationTok}[1]{\textcolor[rgb]{0.56,0.35,0.01}{\textbf{\textit{#1}}}}
\newcommand{\KeywordTok}[1]{\textcolor[rgb]{0.13,0.29,0.53}{\textbf{#1}}}
\newcommand{\NormalTok}[1]{#1}
\newcommand{\OperatorTok}[1]{\textcolor[rgb]{0.81,0.36,0.00}{\textbf{#1}}}
\newcommand{\OtherTok}[1]{\textcolor[rgb]{0.56,0.35,0.01}{#1}}
\newcommand{\PreprocessorTok}[1]{\textcolor[rgb]{0.56,0.35,0.01}{\textit{#1}}}
\newcommand{\RegionMarkerTok}[1]{#1}
\newcommand{\SpecialCharTok}[1]{\textcolor[rgb]{0.00,0.00,0.00}{#1}}
\newcommand{\SpecialStringTok}[1]{\textcolor[rgb]{0.31,0.60,0.02}{#1}}
\newcommand{\StringTok}[1]{\textcolor[rgb]{0.31,0.60,0.02}{#1}}
\newcommand{\VariableTok}[1]{\textcolor[rgb]{0.00,0.00,0.00}{#1}}
\newcommand{\VerbatimStringTok}[1]{\textcolor[rgb]{0.31,0.60,0.02}{#1}}
\newcommand{\WarningTok}[1]{\textcolor[rgb]{0.56,0.35,0.01}{\textbf{\textit{#1}}}}
\usepackage{longtable}
\ifxetex
  \usepackage[setpagesize=false, % page size defined by xetex
              unicode=false, % unicode breaks when used with xetex
              xetex]{hyperref}
\else
  \usepackage[unicode=true]{hyperref}
\fi
\hypersetup{breaklinks=true,
            bookmarks=true,
            pdfauthor={},
            pdftitle={The Effect of Transit Signal Priority on Headway Adherence for Bus Rapid Transit},
            colorlinks=false,
            urlcolor=blue,
            linkcolor=magenta,
            pdfborder={0 0 0}}
\urlstyle{same}  % don't use monospace font for urls

\setcounter{secnumdepth}{5}
% Pandoc toggle for numbering sections (defaults to be off)


% Pandoc header
\usepackage{booktabs}



\begin{document}
\begin{frontmatter}

  \title{The Effect of Transit Signal Priority on Headway Adherence for Bus Rapid Transit}
    \author[Brigham Young University]{Gregory Macfarlane\corref{1}}
   \ead{gregmacfarlane@byu.edu} 
    \author[Brigham Young University]{Grant Schultz}
   \ead{gschultz@byu.edu} 
    \author[WCEC]{Michael Sheffield}
   \ead{cat@example.com} 
    \author[Brigham Young University]{Logan Bennett}
   \ead{cat@example.com} 
      \address[Brigham Young University]{Civil and Environmental Engineering Department, 430 Engineering Building, Provo, Utah 84602}
    \address[WCEC]{Some other place}
      \cortext[1]{Corresponding Author}
  
  \begin{abstract}
  This is where the abstract should go.
  \end{abstract}
  
 \end{frontmatter}

\hypertarget{intro}{%
\section{Questions}\label{intro}}

Transit signal priority (TSP) is a technology that allows traffic signals to
change their timing plans to accommodate transit vehicles. This may involve
extending a green phase until the vehicle passes, triggering an early green if
there is a vehicle waiting at the light, or even running specific transit-only
phases if the transit vehicles must make a turn across the automobile traffic lanes.
TSP is an important tool in helping transit vehicles maintain on-time
performance {[}citations{]}. Often, TSP will only be triggered at a specific signal
if the vehicle is running some amount behind its schedule, thus maximizing green
time for automobiles and minimizing automobile delay when the bus is otherwise
on schedule.

In high-frequency transit systems where the goal is not schedule adherence but
rather \emph{headway} adherence, the specific vehicle schedule might not be published
for transit riders. The specific vehicles might follow an internal schedule, but
it is not clear whether headway adherence can be improved with a schedule-based TSP
regime. The research questions are therefore:

\begin{itemize}
\tightlist
\item
  Does TSP improve headway adherence for rapid transit systems?
\item
  Does the improvement change by time of day or for particular portions of
  a rapid transit route?
\item
  Is there an average improvement, or is there a reduction only in extreme delay?
\end{itemize}

\hypertarget{methods}{%
\section{Methods}\label{methods}}

\begin{Shaded}
\begin{Highlighting}[]
\KeywordTok{library}\NormalTok{(tidyverse)}
\end{Highlighting}
\end{Shaded}

\hypertarget{data}{%
\subsection{Data}\label{data}}

\begin{Shaded}
\begin{Highlighting}[]
\CommentTok{# this dataset was constructed from raw data using the script in R/datamaker.R}
\CommentTok{# the processed data cannot be stored because it is too big for git}
\NormalTok{uvx_time_points <-}\StringTok{ }\KeywordTok{read_rds}\NormalTok{(}\StringTok{"data/uvx_timepoints.rds"}\NormalTok{)}
\end{Highlighting}
\end{Shaded}

In 2018, the Utah Transit Authority (UTA) launched a Bus Rapid Transit (BRT)
system in Provo, Utah, United States. Known as the Utah Valley eXpress (UVX),
the system was highly successful prior to the onset of the COVID-19 pandemic
with more than 10,000 riders per average weekday {[}cite{]}. The system connects
two commuter rail stations, two major universities (Brigham Young and Utah Valley),
and commercial retail districts in Orem and Provo. The system includes X miles of
dedicated lanes on its X-mile route, and 18 stations with a mix of center and side
boarding, all at level floors. The system has been free for all riders since its
opening.

Bus Rapid Transit (BRT) systems are increasingly used by cities around the world
due to their ability to address high passenger demand without exorbitant upfront
costs. The cost savings come from the fact that BRTs operate on Right-of-Way
category B (ROW-B) facilities with separated designated lanes that are subject
to traffic control at intersections. In this way, buses have an unimpeded
corridor on which to complete trips. However, the subjection to traffic control
creates variability in the operating times of these systems, and several
strategies have been developed and implemented to increase their reliability and
operating times.One such strategy is the use of Transit Signal Priority (TSP),
which alters priority at signalized intersections to provide for reduced running
times between stops and a more even distribution of headways. In general, TSP
seeks to improve operations by increasing the efficiency with which buses
navigate signalized intersections, but it can be distinguished by a few classes
\citep{DELGADO201528}:

1.Active or passive priority. Active priority involves timing adjustments made
according to real-time data. Passive priority involves offline measures, such
as optimized cycle lengths, based off of historical data.

2.Total, partial, or relative priority. In total priority, control actions such
as phase jumping, phase insertion, green extension, or early termination of
the red seek to create zero delay for buses. Partial priority limits control
actions to those that provide less interruptions to other traffic, such as
green extensions and early termination of the red. In relative priority, buses
compete with general traffic at lights for priority.

3.Unconditional or conditional priority. Unconditional priority means buses
receive priority at all times whereas conditional priority only provides buses
with needed control measures when the buses are late.

Adaptive signal priority involves real-time control adjustments made in order to
optimize the throughput of both buses and general traffic,in which the delay of
each is considered and a controller decides on a response most pertinent to the
current traffic conditions. \citet{NI20201} compares adaptive signal priority with
active and passive priority, whereas \citet{ALDEEK2017227} groups it with unconditional
and conditional priority.

Many studies have been done to determine the effects of various TSP strategies
and configurations. In general, TSP strategies have been found to improve
performance of transit systems such as BRT in a number of ways, including
reduced delay, improved reliability, and mitigated effects on general traffic.

\citet{ISHAQ2020946} used a set of trip data from a BRT system in Haifa, Israel to
study the relation between service reliability, fleet management, and service
utilization. They found that full and unconditional traffic signal priority
given at all signals led to an 18\% reduction in total vehicle trip time and
contributed to a 60\% reduction in the standard deviation of trip time.
Similarly, \citet{ALDEEK2017227} found that compared to several other TSP strategies,
BRT with unconditional TSP provided the best travel time, speed, number of
stops, and delay enhancements but resulted in significant crossing street
delays, especially at major intersections with high traffic demand. While the
unconditionalTSP strategy was found effective in termsof the delay of the BRT
system, concerns rise over the effects on side-street traffic if unconditional
priority consistently gives right of way to the buses. \citet{Liu2018} clarified that
``signal priorities are provided more efficiently and on a more informed basis,
with fewer impacts on other traffic operations than the use of unconditional
TSP.''The implementation of unconditional TSP may only be practical where
crossing street volumes are low, and must reasonably be done in conjunction with
studies of how non-transit traffic is effected, as each system's corridors have
differing needs based off of demand and geometry.

\citet{Liu2018} performed a study using transit operation data from a bus route found
in Salt Lake County, Utah. A microscopic simulation was used to test GPS-based
TSP scenarios. GPS-based TSP uses a GPS to achieve real-time (active) bus
locating and advanced wireless communication technologies to achieve
comprehensive analysis of operating information. Then a data-driven
optimization method was implemented to understand the effects of flexible
granting of signal priority across several models. They found that overall,
BRT travel time decreased in all models where TSP was employed when compared
to the base BRT model. \citet{ALDEEK2017227} also used simulation models to compare
several scenarios. Using field data from a corridor in Orlando, Florida, they
found that BRT with conditional TSP that was engaged when the buses were 3
minutes behind experienced significantly improved travel times, average speed,
and average total delay per vehicle, with minor effects on crossing street
delays, when compared with BRT systems with no TSP or 5-minute conditional
thresholds.These results indicate that TSP methods are effective in reducing
delay. However, they ignore the important aspect of system reliability, which
is critical to improving running times and maintaining ridership.

\citet{YANG20151} performed a study of a pre-detective TSP strategy for BRT with active
priority coordination between primary and secondary intersections using data
from Changzhou, China. Using microscopic simulation, they tested three
scenarios (traditional TSP, pre-detective TSP, and pre-detective TSP with
coordination) against a no TSP base scenario. They found that pre-detective
signal priority with coordination was most effective, with bus intersection
delay decreasing by 67.4\% and headway adherence improving by about 40\% when
compared with a no TSP strategy, while reducing normal traffic delay.

\citet{DELGADO201528} studied station and interstation control jointly to determine an
optimal control strategy for a single-service transit corridor applied to a
high-frequency transit system, with the goal of evaluating effects over the
whole corridor. A strategy of green extension with holding buses at stops was
shown to produce a greater reduction in waiting times for users, as well as
reducing variability and improving headway adherence.

\citet{NI20201} performed a study of passive TSP control for a BRT system in Taichung,
Taiwan. Microsimulation incorporated with a genetic algorithm was developed to
coordinate signal offsets along an arterial with BRT operations, with the
purpose of balancing improvements in BRT system delays and changes to LOS of
other traffic. They found that passive TSP control can reduce approximately
22\% of transit delay, with the smallest delay experienced in conjunction with
a strategy of a three-minute scheduled headway. Additionally, the 3-minute
departure headway scenario exhibited the best service reliability among the
scenarios, with a headway stability near 100\%.

These studies indicate the need to view BRT system performance in terms of both
system delay and reliability. Especially when viewed across an entire system, it
is possible for running times to remain consistent while reliability of
individual buses is low. In particular, a bunching phenomenon may occur from a
positive feedback loop of fluctuations in passenger demand and traffic
conditions (\citet{DELGADO201528}).In scenarios where bunching is a problem, it may not
be sufficient to base control strategies off of a schedule, rather on reliable
operations. This may prove difficult in implementation. For instance,
\citet{ISHAQ2020946} noted that the procurement agreement between the public transport
authority in Israel and the incumbent operator implies that a schedule-based
control strategy is required, as opposed to a headway-based (reliability-based)
control strategy. However, when TSP control strategies are reliability-based,
measures of reliability increase and the bunching phenomenon can be reduced.

\citet{CATS2014223} performed a study of a regularity-driven operation scheme in
Stockholm, Sweden to improve reliability and mitigate bus bunching. Using a
series of field experiments and measuring performance with regularity
indicators such as headway coefficient of variation, headway adherence, and
average excess waiting time, he determined that a headway-based control
strategy was effective in not only narrowing the headway distribution from the
previous schedule-bases strategy, but also reducing waiting times, more evenly
distributing dwell times, and maintaining running times. \citet{CHEN2012} also
performed a headway-based study to measure the effects of real-time preventive
operations control. They found that reliability is improved with lower
permitted headway deviations and lower fluctuations of running times. When
real-time information is used to predict service reliability and trigger
preventive control, such as reduced dwell time and speed adjustments, bus
bunching can be reduced.

Both of these studies effectively demonstrate that solutions to bunching and
other reliability-based operational problems are most effectively addressed
through reliability-based control strategies. However, little has been done to
evaluate the relationship between methods of TSP in a reliability-based control
strategy. A study of TSP strategies with a focus on headway distribution could
prove useful in gauging the benefits of TSP in a headway-based system, such as
the UVX BRT system in Utah Valley.

\hypertarget{findings}{%
\section{Findings}\label{findings}}

\bibliography{book.bib}


\end{document}


